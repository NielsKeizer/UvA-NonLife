\documentclass[11pt]{article}

\usepackage[english]{babel}                 %% hyphenation rules, spell-checker
\usepackage{amsmath,amssymb}                        %% macros like align* and pmatrix
\usepackage{graphicx,epstopdf}              %% for .eps graphs
\usepackage[official]{eurosym}              %% 1 \euro
\usepackage[a4paper,margin=2cm]{geometry}   %% margins
\usepackage{hyperref}                       %% hyperlinks to urls
\usepackage{float}                    

\frenchspacing                              %% no extra space after period
\addtolength{\parskip}{0.5\baselineskip}    %% some white space between paragraphs
\setlength{\parindent}{0pt}                 %% but no indentation
\renewcommand{\baselinestretch}{1.1}        %% line spacing of TeX is small
\DeclareMathOperator{\E}{\mathbb{E}}
\DeclareMathOperator{\Var}{\text{Var}}
\DeclareMathOperator{\Cov}{\text{Cov}}


\title{Non-life --- Assignment NL2}  %% don't forget to change!

\author{
  Niels Keizer\footnote{Student number: 10910492}
  \quad and \quad
  Robert Jan Sopers\footnote{Student number: 0629049}
}

\date{\today}

\begin{document}

\maketitle

\section{Simulating an insurance portfolio-App. A3}

\subsection*{Q1}

How many bytes does it take to store $1,\ldots, 10, 1000, 100000$ logical values \verb|TRUE|/\verb|FALSE|?

We assume that $1,\ldots, 10$ means all the integers from 1 to 10. To how many bytes are needed in \verb|R|, we use the function \verb|object.size()|.

\begin{verbatim}
> for (n_values in c(1,2,3,4,5,6,7,8,9,10,1000,100000)){
+   hh <- rep(TRUE,n_values)
+   rr <- sample(c(TRUE,FALSE),n_values,repl=TRUE,prob=c(1,1))
+   af <- as.factor(rr)
+   print(c(n_values, object.size(hh), object.size(rr), object.size(af)))
+ }
[1]   1  48  48 464
[1]   2  48  48 464
[1]   3  56  56 528
[1]   4  56  56 528
[1]   5  72  72 544
[1]   6  72  72 488
[1]   7  72  72 544
[1]   8  72  72 544
[1]   9  88  88 560
[1]  10  88  88 560
[1] 1000 4040 4040 4512
[1] 100000 400040 400040 400512
\end{verbatim}

The first column of the output is the length of the vector. The second column indicates the size in bytes of a vector filled with only \verb|TRUE| values. The third with a random selection of \verb|TRUE| and \verb|FALSE|. The final column represents the size of the randomized vector, after it has been turned into a factor object.

\subsection*{Q2}

To obtain the \verb|y| vector, we first need to run the following code:

\begin{verbatim}
> n.obs <- 10000; set.seed(4)
> # n.obs <- 10000; set.seed(4) # Gebruik deze regel voor een grotere sample size.
> sx <- as.factor(sample(1:2, n.obs, repl=TRUE, prob=c(6,4)))
> jb <- as.factor(sample(1:3, n.obs, repl=TRUE, prob=c(3,2,1)))
> re.tp <- sample(1:9, n.obs, repl=TRUE, prob=c(.1,.05,.15,.15,.1,.05,.1,.1,.2))
> tp <- as.factor(c(1,2,3,1,2,3,1,2,3)[re.tp]) 
> re <- as.factor(c(1,1,1,2,2,2,3,3,3)[re.tp])
> mo <- 3 * sample(1:4, n.obs, repl=TRUE, prob=c(1,1,0,8))
> mu <- 0.05 * c(1,1.2)[sx] *
+              c(1,1,1)[jb] * 
+              c(1,1.2,1.44)[re] * 
+              1.2^(0:2)[tp] * mo/12 
> y <- rpois(n.obs, mu)
> table(y)
y
   0    1    2    3 
9276  702   20    2 
\end{verbatim}

Which is then inspected by calculating \verb|mean(y)|, \verb|var(y)| and the overdispersion factor \verb|var(y)|/\verb|mean(y)|.

\begin{verbatim}
> cbind(mean=mean(y),variance=var(y),phi=var(y)/mean(y))
       mean  variance       phi
[1,] 0.0748 0.0744124 0.9948182
\end{verbatim}

The overdispersion factor is smaller than $1$. This is possible because we are looking at a relatively small sample, with low probabilities. If we would take a much larger sample, the value would be larger than $1$. We check this by running the same code, but with a sample 100 times larger. This gives a result with an overdispersion factor larger than $1$.

\begin{verbatim}
> table(y)
y
     0      1      2      3      4 
931128  66053   2734     82      3 
> cbind(mean=mean(y),variance=var(y),phi=var(y)/mean(y))
         mean   variance      phi
[1,] 0.071779 0.07262285 1.011756
\end{verbatim}

\subsection*{Q3}

We create a dataframe by using the function \verb|aggregate()|.

\begin{verbatim}
> aggr <- aggregate(list(Expo=mo/12,nCl=y,nPol=1), list(Jb=jb,Tp=tp,Re=re,Sx=sx), sum)
\end{verbatim}

Then we compare the sizes.

\begin{verbatim}
> object.size(aggr) 
5336 bytes
> object.size(mo) 
80040 bytes
> object.size(y) 
40040 bytes
> object.size(jb) + object.size(tp) + object.size(re) + object.size(sx)
162240 bytes
\end{verbatim}

The amount of memory gained is equal to $80040 + 40040 + 162240 - 5336 = 276984$ bytes.

\subsection*{Q4}

According to MART Sec. 3.9.3, the maximum likelihood estimate $\hat{\lambda}_{3,3,3,2}$ is equal to the number of claims divided by the exposure.

\begin{verbatim}
> aggr[54,]
   Jb Tp Re Sx   Expo nCl nPol
54  3  3  3  2 115.75  13  130
> lambda3332 <- aggr$nCl[54]/aggr$Expo[54]
> lambda3332
[1] 0.112311
\end{verbatim}

In the first command, we show that observation 54 contains the desired aggregated values to calculate the estimate, which is then determined at $0.112$.

\section{Exploring the automobile portfolio of Sec. 9.5}

First we execute the following code in \verb|R| to generate the portfolio.

\begin{verbatim}
> rm(list=ls(all=TRUE)) 
> n <- scan(n=54) ## read 54 numbers into vector n
1:   1  8 10  8  5 11 14 12 11 10  5 12 13 12 15 13 12 24
19: 12 11  6  8 16 19 28 11 14  4 12  8 18  3 17  6 11 18
37: 12  3 10 18 10 13 12 31 16 16 13 14  8 19 20  9 23 27
Read 54 items
> expo <- scan(n=54) ## the number of policies 
1:  10 22 30 11 15 20 25 25 23 28 19 22 19 21 19 16 18 29
19: 25 18 20 13 26 21 27 14 16 11 23 26 29 13 26 13 17 27
37: 20 18 20 29 27 24 23 26 18 25 17 29 11 24 16 11 22 29
Read 54 items
> expo <- 7 * expo ## each policy is in force during a 7-year period
> sex <- gl(2,27); region <- gl(3, 9, 54); type <- gl(3, 3, 54); job <- gl(3, 1, 54)
\end{verbatim}

\subsection*{Q5}

We are asked to comment on the difference between to lines of \verb|R| code.

\begin{verbatim}
> str(type)
Factor w/ 3 levels "1","2","3": 1 1 1 2 2 2 3 3 3 1 ...
> str(rep(1:3, each=3, len=54))
int [1:54] 1 1 1 2 2 2 3 3 3 1 ...
\end{verbatim}

The \verb|str()| function compactly displays the structure of an arbitrary R object.
\verb|type| contains a \verb|Factor| object, with 3 ordered levels (or categories), and a list of integers which indicate which element is at that position.
\verb|rep(1:3, each=3, len=54)| creates a vector of integers of three ones, three twos and three threes, repeated to a length of 54. Both objects


\end{document}
