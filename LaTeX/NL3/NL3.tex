\documentclass[11pt]{article}

\usepackage[english]{babel}                 %% hyphenation rules, spell-checker
\usepackage{amsmath,amssymb}                        %% macros like align* and pmatrix
\usepackage{graphicx,epstopdf}              %% for .eps graphs
\usepackage[official]{eurosym}              %% 1 \euro
\usepackage[a4paper,margin=2cm]{geometry}   %% margins
\usepackage{hyperref}                       %% hyperlinks to urls
\usepackage{float}                    

\frenchspacing                              %% no extra space after period
\addtolength{\parskip}{0.5\baselineskip}    %% some white space between paragraphs
\setlength{\parindent}{0pt}                 %% but no indentation
\renewcommand{\baselinestretch}{1.1}        %% line spacing of TeX is small
\DeclareMathOperator{\E}{\mathbb{E}}
\DeclareMathOperator{\Var}{\text{Var}}
\DeclareMathOperator{\Cov}{\text{Cov}}


\title{Non-life --- Assignment NL3}  %% don't forget to change!

\author{
  Niels Keizer\footnote{Student number: 10910492}
  \quad and \quad
  Robert Jan Sopers\footnote{Student number: 0629049}
}

\date{\today}

\begin{document}

\maketitle

\section*{Analyzing a bonus-malus system using GLM}

\subsection*{Q1}

\subsubsection*{a)}
We are asked to check if the values in Table 9.8 from MART are correct. For this, we first run the code given in the exercise:

\begin{verbatim}
> rm(list=ls(all=TRUE)) ## First remove traces of previous sessions
> fn <- "http://www1.fee.uva.nl/ke/act/people/kaas/Cars.txt"
> Cars <- read.table(fn, header=TRUE)
> Bminus1 <- Cars$B - 1; Bis14 <- as.numeric(Cars$B==14)
> Cars$A <- as.factor(Cars$A); Cars$R <- as.factor(Cars$R)
> Cars$M <- as.factor(Cars$M); Cars$U <- as.factor(Cars$U)
> Cars$B <- as.factor(Cars$B); Cars$WW <- as.factor(Cars$WW)
> ActualWt <- c(650,750,825,875,925,975,1025,1075,1175,1375,1600)
> W <- log(ActualWt/650)[Cars$WW]
> 
> # GLM analysis
> 
> g1 <- glm(TotCl/Expo~R+A+U+W+Bminus1+Bis14, quasipoisson, wei=Expo, data=Cars)
> g2 <- glm(TotCl/Expo~R+A+U+W+Bminus1+Bis14+M, quasipoisson, wei=Expo, data=Cars)
> g3 <- glm(TotCl/Expo~R+A+U+W+B, quasipoisson, wei=Expo, data=Cars)
> 
> anova(g1,g2)
Analysis of Deviance Table

Model 1: TotCl/Expo ~ R + A + U + W + Bminus1 + Bis14
Model 2: TotCl/Expo ~ R + A + U + W + Bminus1 + Bis14 + M
  Resid. Df Resid. Dev Df Deviance
1      7515   38616941            
2      7513   38614965  2   1975.8
> anova(g1,g3)
Analysis of Deviance Table

Model 1: TotCl/Expo ~ R + A + U + W + Bminus1 + Bis14
Model 2: TotCl/Expo ~ R + A + U + W + B
  Resid. Df Resid. Dev Df Deviance
1      7515   38616941            
2      7504   38544506 11    72435
> 
> # Multiplicative coefficients
> options(digits=7)
> exp(coef(g1)); exp(coef(g2)); exp(coef(g3))
(Intercept)          R2          R3          A2          A3          U2           W
524.3016583   1.0842682   1.1916130   0.4147224   0.6184468   1.3841303   2.3722083
    Bminus1       Bis14 
  0.8978647   1.1053665 
(Intercept)          R2          R3          A2          A3          U2           W 
522.6627527   1.0842767   1.1914111   0.4147232   0.6184538   1.3835062   2.3721668 
    Bminus1       Bis14          M2          M3 
  0.8978640   1.1053568   1.0073260   1.0014581   
(Intercept)          R2          R3          A2          A3          U2           W 
515.5320549   1.0843018   1.1916593   0.4143437   0.6178700   1.3841612   2.3722369
         B2          B3          B4          B5          B6          B7          B8 
  0.9111279   0.8275175   0.7403718   0.6842609   0.6088526   0.5416103   0.4489065 
         B9         B10         B11         B12         B13         B14 
  0.4151901   0.3888576   0.3459030   0.3143452   0.2832722   0.2773037 
\end{verbatim}

All coefficients can be checked individually against table 9.8 and are the same, except for models \verb|g1| and \verb|g2|, because the bonus malus risk factor is taken as numeric. This means that the factors in the table have been calculated from the factor for $B2$ to the power $B - 1$. The coefficient for \verb|Bminus1| differs only in the 7th decimal spot and the table is given with 4 decimals. This means that if we only need to check one of the two models. We do this by recalculating the values in \verb|R|.

\begin{verbatim}
> bm_class <- seq(1,13,1)
> bm_coef <- exp((bm_class-1)*coef(g1)["Bminus1"])
> bm_coef
[1] 1.0000000 0.8978647 0.8061610 0.7238236 0.6498956 0.5835184 0.5239205
[8] 0.4704098 0.4223643 0.3792260 0.3404937 0.3057173 0.2744927
\end{verbatim}

These values also correspond with those in table 9.8

\subsubsection*{b)}
Using the coefficients of \verb|g1|, \verb|g2| and \verb|g3|, compute the fitted values for the cell 4000.

For this, we use the coefficients in \verb|R|. Recalculating these by hand would be rather pointless and is an exercise in working neatly over understanding the subject matter.

\begin{verbatim}
> # Observed value
> g1$y[4000]
4000 
326.4545 
> 
> # Fitted value
> fitted(g1)[4000]; fitted(g2)[4000]; fitted(g3)[4000]
4000 
634.0642 
4000 
636.416 
4000 
644.5283 
\end{verbatim}

What we can see is the all three GLM's have a fitted value that is about twice as large as the actual value. Not one of the models is close to the observed value, but the models are quite close together in their estimate.

\subsubsection*{c)}

We now explain the result of the following \verb|R|-code.

\begin{verbatim}
> g2$family$linkinv(model.matrix(g2)[4000,]%*%coef(g2))
        [,1]
[1,] 636.416
\end{verbatim}

This result is equal to the fitted value of the \verb|g2| model. This is no surprise, considering the code is equal to the definition of the fitted value for cell 4000. The inner product of the values of the risk factors and their corresponding coefficients gives the linear estimator for that cell, after which the \verb|linkinv| function is applied, which is the exponential function. This results in the fitted value.

\subsection*{Q2}

First we will determine the scale factor $\phi$ using a 'rich' model, meaning that the values of both the weight of the car and the BM class are used as factors.

\begin{verbatim}
> g.rich <- glm(TotCl/Expo~R+A+U+WW+B, quasipoisson, wei=Expo, data=Cars)
> anova(g.rich)
Analysis of Deviance Table

Model: quasipoisson, link: log

Response: TotCl/Expo

Terms added sequentially (first to last)


     Df Deviance Resid. Df Resid. Dev
NULL                  7523  116167018
R     2  2586478      7521  113580540
A     2 23288859      7519   90291681
U     1  4479946      7518   85811735
WW   10  6931993      7508   78879742
B    13 40358336      7495   38521406
\end{verbatim}

We determine the scale factor to be $\frac{38521406}{7495} = 5139.61$. Or in \verb|R|:

\begin{verbatim}
> phi <- 38521406/7495
\end{verbatim}

To check whether \verb|Bis14| can be removed from the model, we use an \verb|anova| call on the model with \verb|Bis14| (\verb|g1|) and without (\verb|g.test|).

\begin{verbatim}
> g.test <- glm(TotCl/Expo~R+A+U+W+Bminus1, quasipoisson, wei=Expo, data=Cars)
> anova(g.test,g1)
Analysis of Deviance Table

Model 1: TotCl/Expo ~ R + A + U + W + Bminus1
Model 2: TotCl/Expo ~ R + A + U + W + Bminus1 + Bis14
  Resid. Df Resid. Dev Df Deviance
1      7516   38755743            
2      7515   38616941  1   138802
\end{verbatim} 

Next we test if the inclusion of \verb|Bis14| is significant:

\begin{verbatim}
> test <- function (Df, Deviance){
+   scaled.dev <- Deviance/phi
+   test.dev <- qchisq(0.95,Df)
+   return(scaled.dev>test.dev)
+ }
> test(1, 138802)
[1] TRUE
\end{verbatim}

First we calculate the scaled deviance. Then we calculate the 95-th percentile of the $\chi^{2}(k)$ distribution with \verb|Df| degrees of freedom. When the improvement of scaled deviance is larger than the test value, the increase is significant. The test is implemented as a function, so it can be reused in the rest of the exercise. Also the test returns \verb|TRUE|, therefore the inclusion of \verb|Bis14| is a significant improvement of the model and can not be removed.

Then we check if \verb|B| can be removed from model \verb|g3|.

\begin{verbatim}
> g.test <- glm(TotCl/Expo~R+A+U+W, quasipoisson, wei=Expo, data=Cars)
> anova(g.test,g3)
Analysis of Deviance Table

Model 1: TotCl/Expo ~ R + A + U + W
Model 2: TotCl/Expo ~ R + A + U + W + B
  Resid. Df Resid. Dev Df Deviance
1      7517   78902891            
2      7504   38544506 13 40358385
> test(13,40358385)
[1] TRUE
\end{verbatim}

The test value is \verb|TRUE|, so \verb|B| can not be removed from the model. Next we check whether \verb|W| can be removed from the model:

\begin{verbatim}
> g.test <- glm(TotCl/Expo~R+A+U+B, quasipoisson, wei=Expo, data=Cars)
> anova(g.test,g3)
Analysis of Deviance Table

Model 1: TotCl/Expo ~ R + A + U + B
Model 2: TotCl/Expo ~ R + A + U + W + B
  Resid. Df Resid. Dev Df Deviance
1      7505   45495122            
2      7504   38544506  1  6950616
> test(1,6950616)
[1] TRUE
\end{verbatim}

This result implies that \verb|W| can not be removed from the model \verb|g3|.

Is it helpful to allow separate coefficients for the weight class in model \verb|g1|. We again check using the \verb|anova| and \verb|test| functions.

\begin{verbatim}
> g.test <- glm(TotCl/Expo~R+A+U+WW+Bminus1+Bis14, quasipoisson, wei=Expo, data=Cars)
> anova(g1, g.test)
Analysis of Deviance Table

Model 1: TotCl/Expo ~ R + A + U + W + Bminus1 + Bis14
Model 2: TotCl/Expo ~ R + A + U + WW + Bminus1 + Bis14
  Resid. Df Resid. Dev Df Deviance
1      7515   38616941            
2      7506   38593888  9    23053
> test(9,23053)
[1] FALSE
\end{verbatim}

This shows that allowing separate coefficients for the weight classes would not be an improvement.

\end{document}
